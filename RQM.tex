\documentclass[11pt]{article}
\usepackage[margin=1in]{geometry}
\usepackage{jheppub} % for details on the use of the package, please see the JINST-author-manual
% page formatting
\usepackage{fancyhdr}
\pagestyle{fancy}

\renewcommand{\sectionmark}[1]{\markright{\textsf{#1}}}
\renewcommand{\subsectionmark}[1]{}
\lhead{\textbf{\thepage} \ \ \nouppercase{\rightmark}}
\chead{}
\rhead{}
\lfoot{}
\cfoot{}
\rfoot{}
\setlength{\headheight}{14pt}

\linespread{1.03} % give a little extra room
\setlength{\parindent}{0.2in} % reduce paragraph indent a bit
\setcounter{secnumdepth}{2} % no numbered subsubsections
\setcounter{tocdepth}{2} % no subsubsections in ToC

\usepackage{lineno}
\usepackage{amsmath,amsthm,amsfonts,amssymb,amscd,physics,cancel,mathtools}
\usepackage{tcolorbox}
\usepackage{marginnote,tensor}
\usepackage[spanish]{babel}
%~~~~~~~~~ Document setup
% \usepackage[spanish]{babel} % English formatting
\usepackage[utf8]{inputenc} % Standard encoding
% \usepackage[a4paper,left=3cm,bottom=3cm]{geometry} % Page formatting
\usepackage{indentfirst} % Indents the first paragraph
\usepackage{amsmath} % Maths type package
\usepackage{bm} % Bold font maths
\usepackage{graphicx} % Advanced graphics package
\usepackage[export]{adjustbox} 
\usepackage{pdflscape} % Make pages landscape
\usepackage{fancyhdr} % Fancy headers
% \usepackage[colorlinks=true,citecolor=blue,urlcolor=blue,linkcolor=black]{hyperref} % Link colours
%\usepackage{natbib} % Bibliography
% \usepackage{flafter} % Reference any 'float'
% \usepackage[framemethod=tikz]{mdframed} % Box off stuff
\usepackage{color} % Colour support
\usepackage{wrapfig} % Text flowing around figures
\usepackage{lipsum} % Generates meaningless text
\usepackage{xcolor}
%\usepackage{biblatex}
%\usepackage[backend=bibtex]{biblatex}
%\addbibresource{bibliography.bib}
%\hypersetup{colorlinks=true, linkcolor=blue}


\theoremstyle{definition}
\newtheorem{ej}{Ejemplo}[section]
\newtheorem{sol}{Solución}[section]
\newtheorem{prop}{Propiedad}[section]
\newtheorem{dem}{Proof}[section]
\newtheorem{defi}{Definición}[section]
\newtheorem{teor}{Teorema}[section]
\newtheorem{prueba}{Prueba}[section]
\newtheorem{obs}{Observación}[section]


\newcommand{\Lag}{L=L(q_i,\dot{q}_i,t)}
\def\anillo{$(A,+,\cdot)$}
\def\campo{$(K,+,\cdot)$}
\def\espvec{$(M,K,\bullet)$}
\def\algebra{$(A,K,\bullet)$}
\def\qp{q'}
\def\tp{t'}
\def\qd{\dot{q}}
\def\qdd{\ddot{q}}
\def\qdp{\dot{q}'}
\def\dttp{\dv{t}{t'}}
\def\dtpt{\dv{t'}{t}}
\def\db{\bar{\delta}}
\def\l({\left(}
\def\r){\right)}
\def\EL{\pdv{L}{q_i}-\dv{t}\pdv{L}{\qd_i}}
\def\uin{u(x)_{i\nu}}
\def\he{\hat{e}}
\def\M{\mathcal{M}}
\def\vec{\vb*}
\def\Lmn{\Lambda^\m_{~\n }}
\def\omn{\omega^\m_{~\n }}
\def\id{\mathbb{I}}
\def\L{\mathcal{L}}
\def\xt{(\vec{x},t)}
\def\rt{(\vec{r},t)}
\def\xpt{(\vec{x}',t)}
\def\xppt{(\vec{x}'',t)}
\def\dxxp{\delta^{(3)}(\vec{x}-\vec{x}')}
\def\dxxpp{\delta^{(3)}(\vec{x}-\vec{x}'')}


\def\a{\alpha}
\def\b{\beta}
\def\g{\gamma}
\def\G{\Gamma}
\def\d{\delta}
%\def\D{\Delta}
%\def\e{\eta}
\def\la{\lambda}
\def\La{\Lambda}
\def\k{\kappa}
\def\m{\mu}
\def\n{\nu}
%\def\r{\rho}
%\def\p{\rho}
\def\o{\omega}
\def\s{\sigma}
\def\S{\Sigma}
\def\t{\tau}
\def\p{\pi}
\def\f{\phi}
\def\vf{\varphi}
\def\ep{\epsilon}
\def\th{\theta}
\def\Th{\Theta}
\def\z{\zeta}


%-----COLORS LIST ------
\definecolor{azure(colorwheel)}{rgb}{0.0, 0.5, 1.0}
\definecolor{DarkViolet}{RGB}{148,0,211}
\definecolor{myDarkBlue}{rgb}{0,0.1,0.7}
\definecolor{DarkBlue}{RGB}{0,0,153}
\definecolor{amber}{rgb}{1.0, 0.49, 0.0}
\definecolor{amaranth}{rgb}{0.9, 0.17, 0.31}
\definecolor{nicered}{rgb}{0.7,0.1,0.1}
\definecolor{brown}{rgb}{0.5,0.1,0.1}
\definecolor{nicegreen}{rgb}{0.0,0.3,0.0}
\definecolor{tealgreen}{rgb}{0.0, 0.51, 0.5}
\def\red#1{{\color{red} #1}}
\def\green#1{{\color{green} #1}}
\def\blue#1{{\color{blue} #1}}
\def\orange#1{{\color{orange} #1}}
%----------------------
\newcommand{\mycolor}{DarkViolet}
\def\myColor#1{{\color{\mycolor} #1}}
\definecolor{tclr}{RGB}{148,0,211}
%----------------------
\newcommand{\corr}[1]{\textcolor{nicered}{#1}}
\newcommand{\nick}[1]{\textcolor{olive}{#1}}
\newcommand{\teo}[1]{\textcolor{azure(colorwheel)}{#1}}
\newcommand{\chteo}[2]{\corr{\st{#1}} \teo{(#2)}}
\newcommand{\bako}[1]{\textcolor{DarkViolet}{#1}}
\newcommand{\than}[1]{\textcolor{magenta}{#1}}

%----------------------
\usepackage{hyperref}
\hypersetup{colorlinks,bookmarksopen,
	bookmarksnumbered,
	citecolor={nicered},
	linkcolor={myDarkBlue},
	urlcolor={tealgreen},
	pdfstartview=FitH}




% \arxivnumber{1234.56789} % if you have one

%\title{\boldmath Teoría Clásica de Campos}

% Collaborations

%% [A] If main author
%% \collaboration{\includegraphics[height=17mm]{collabroation-logo}\\[6pt]
%%  XXX collaboration}

%% or
%% [B] If "on behalf of"
%% \collaboration[c]{on behalf of XXX collaboration}


% Authors
% The "\note" macro will give a warning: "Ignoring empty anchor...", you can safely ignore it.

%% [A] simple case: 2 authors, same institution
%% \author[1]{A. Uthor\note{Corresponding author.}}
%% \author{and A. Nother Author}
%% \affiliation{Institution,\\Address, Country}

%% or, e.g.
%% [B] more complex case: 4 authors, 3 institutions, 2 footnotes
%% \author[a,b]{F. Irst,\note{Now at another university}}
%% \author[c]{S. Econd,}
%% \author[a,2]{T. Hird\note{Also at Some University.}}
%% \author[c,2]{and Fourth}
%% \affiliation[a]{Institution_1,\\Address, Country}
%% \affiliation[b]{Institution_2,\\Address, Country}
%% \affiliation[c]{Institution_3,\\Address, Country}

%\author{Borja Diez}
%\affiliation{Universidad Arturo Prat}
%% \affiliation{Another University,\\
%% different-address, Country}
%
%% E-mail addresses: only for the corresponding author
%\emailAdd{borjadiez1014@gmail.com}
%
%\abstract{Estas notas son una suerte de recopilado de las clases Teoría Clásica de Campos dictadas por el Dr. Patricio Salgado el primer semestre del 2024.}




\begin{document}
% make title page
\thispagestyle{empty}
\bigskip \
\vspace{0.1cm}

\begin{center}
{\fontsize{22}{22} \selectfont Notas en}
\vskip 16pt
{\fontsize{36}{36} \selectfont \bf \sffamily Mecánica Cuática Relativista}
\vskip 24pt
{\fontsize{18}{18} \selectfont \rmfamily Borja Diez} 
\vskip 6pt
{\fontsize{14}{14} \selectfont \ttfamily borjadiez1014@gmail.com}
\vskip 24pt
\end{center}

Estas notas de clase están basadas en el libro \textit{Relativistic Quantum Mechanics} de W.Greiner, y han sido escritas con propósito de estudio personal.

Adicionalmente,las notas han sido complementadas con desarrollos de cálculo personal y comentarios sacados de la bibliografía citada al final de este documento. 

% make table of contents
\newpage
\tableofcontents
\newpage

\section{Ecuación de onda relativista para partículas de spin-$0$. Ecuación de Klein-Gordon}

La descripción de un fenómeno a altas energías requiere la investigación de ecuaciones de ondas relativistas, es decir, \textit{ecuaciones que sean invariantes de Lorentz}. La transición desde una descripción no-relativista hacia una relativista implica que varios conceptos de la teoría no-relativista deben ser reinvestigados, en particular:
\begin{enumerate}
	\item Coordenadas espaciales y temporales tienen que ser tratadas por igual en la teoría.
	\item Dado que
	\begin{equation}
  \Delta x\sim \frac{\hbar}{\Delta p}\sim \frac{\hbar }{m_0c},
\end{equation}
una partícula relativista no puede ser localizada con más precisión que $\approx \hbar/m_0c$; de otro modo podría ocurrir creación de pares de partículas debido a $E>2m_0c^2$. Así, la idea de partícula libre solo hace sentido si la partícula no está confinada por vínculos externos a un volúmen menor que aproximadamente la \textit{longitud de onda de Compton} $\lambda_{ c}=\hbar/m_0c $. De otro modo, la partícula automáticamente tiene compañeros debido a la reación de partícula-antipartícula.
\item Si la posición de la partícula es incierta, e.d., si
\begin{equation}
  \Delta x>\frac{\hbar }{m_0c},
\end{equation}
entonces el tiempo es tambien incierto, debido a
\begin{equation}
  \Delta t\sim \frac{\Delta x}{c}>\frac{\hbar}{m_0c^2}.
\end{equation}
En una teoría no-relativista $\Delta t$ puede ser arbitrariamente pequeño, porque $c\to \infty$. De este modo, necesitamos la necesidad de reconsiderar el concepto de densidad de probabilidad
\begin{equation}
  \rho(x,y,z,t),
\end{equation}
el cual describe la probabilidad de encontrar una partícula en un lugar definido $\vec{r}$ a un tiempo fijo $t$.
\item A altas energías (relativista) ocurre creación y aniquilación de pares, usualmente en la forma de creación de pares de partícula-antipartícula. Así, a energías relativistas, la conservación de partículas no es más un supuesto válido. Una teoría relativista debe ser capáz de describir creación de pares, polarización del vacio, conversión de partículas, etc. 
\end{enumerate}

\subsection{La ecuación de Klein-Gordon}
De la mecánica cuántica sabemos que la ecuación de Schrödinger
\begin{equation}
  i\hbar \pdv{\psi}{t}=\left[-\frac{\hbar^2}{2m_0}\nabla^2+V(\vec{x})\right]\psi\xt 
\end{equation}
corresponde a la relación de energía no-relativista en forma de operador,
\begin{equation}
  \hat{E}=\frac{\hat{\vec{p}}^2}{2m_0}+V(\vec{x}),\quad \rm donde
\end{equation}
\begin{equation}
  \hat{E}=i\hbar\pdv{t},\qquad \hat{\vec{p}}=-i\hbar\nabla
\end{equation}
son los operadores de energía y momentum respectivamente. Con el fin de obtener una ecuación de onda relativista, comencemos considernado partículas libres con la relación relativista
\begin{equation}
  p^\m p_\m =\frac{E^2}{c^2}-\vec{p}\cdot\vec{p}=m_0^2c^2.
\end{equation}
Reemplazamos el 4-momentum $p^\m $ por el operador 4-momentum
\begin{align}
  \hat{p}^\m =i\hbar\pdv{x_\m }&=i\hbar\left(\pdv{(ct)},\pdv{x_1},\pdv{x_2},\pdv{x_3}\right)\\
  &=i\hbar\left(\pdv{(ct)},-\pdv{x},-\pdv{y},-\pdv{z}\right)\\
  &=i\hbar\left(\pdv{(ct)},-\nabla\right)\\
  &=(\hat{p}_0,\hat{\vec{p}})
\end{align}
Así, obtenemos la \textbf{ecuación de Klein-Gordon} para partículas libres,
\begin{equation}\label{1.21}
  \boxed{\hat{p}^\m \hat{p}_\m \psi=m_0^2c^2\psi}
\end{equation}
Aquí $m_0$ es la masa en reposo de la partícula y $c$ es la velocidad de la luz en el vacío. Notemos que
\begin{align}
  \hat{p}^\m \hat{p}_\m&=\left[i\hbar\left(\pdv{(ct)},-\nabla\right)\right]\left[i\hbar\left(\pdv{(ct)},\nabla\right)\right]\\
  &=-\hbar^2\left[\pdv{(ct)},-\nabla\right]\left[\pdv{(ct)},\nabla\right]\\
  &=-\hbar^2\left(\frac{1}{c^2}\pdv[2]{t},-\nabla^2\right)\\
  &=-\hbar^2\Box
\end{align}
Así, podemos escribir \eqref{1.21} como
\begin{equation}\label{1.22}
  \left(\Box+\frac{m_0^2c^2}{\hbar^2}\right)\psi=0
\end{equation}
Podemos verificar inmediatamente que covaiancia de Lorentz de la ecuación de Klein-Gordon, dado que $ \hat{p}^\m \hat{p}_\m$ es invariante de Lorentz. Notemos también que \eqref{1.22} es la ecuación de onda clásica incluyendo el \textit{término de masa} $m_0^2c^2/\hbar^2$. Soluciones libres son de la forma
\begin{equation}
  \psi=\exp\left(-i\hbar p_\m x^\m \right)=\exp\left[-\frac{i}{\hbar}\left(p_0x^0-\vec{p}\cdot\vec{x}\right)\right]=\exp\left[\frac{i}{\hbar}(\vec{p}\cdot\vec{x}-Et)\right]
\end{equation}











































% Bibliography

%% [A] Recommended: using JHEP.bst file
%% \bibliographystyle{JHEP}
%% \bibliography{biblio.bib}

%% or
%% [B] Manual formatting (see below)
%% (i) We suggest to always provide author, title and journal data or doi:
%% in short all the informations that clearly identify a document.
%% (ii) please avoid comments such as "For a review'', "For some examples",
%% "and references therein" or move them in the text. In general, please leave only references in the bibliography and move all
%% accessory text in footnotes.
%% (iii) Also, please have only one work for each \bibitem.

\newpage
\bibliographystyle{JHEP}
\bibliography{biblio.bib}
\end{document}
