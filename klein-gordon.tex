\section{Ecuación de onda relativista para partículas de spin-$0$. Ecuación de Klein-Gordon}

La descripción de un fenómeno a altas energías requiere la investigación de ecuaciones de ondas relativistas, es decir, \textit{ecuaciones que sean invariantes de Lorentz}. La transición desde una descripción no-relativista hacia una relativista implica que varios conceptos de la teoría no-relativista deben ser reinvestigados, en particular:
\begin{enumerate}
	\item Coordenadas espaciales y temporales tienen que ser tratadas por igual en la teoría.
	\item Dado que
	\begin{equation}
  \Delta x\sim \frac{\hbar}{\Delta p}\sim \frac{\hbar }{m_0c},
\end{equation}
una partícula relativista no puede ser localizada con más precisión que $\approx \hbar/m_0c$; de otro modo podría ocurrir creación de pares de partículas debido a $E>2m_0c^2$. Así, la idea de partícula libre solo hace sentido si la partícula no está confinada por vínculos externos a un volúmen menor que aproximadamente la \textit{longitud de onda de Compton} $\lambda_{ c}=\hbar/m_0c $. De otro modo, la partícula automáticamente tiene compañeros debido a la reación de partícula-antipartícula.
\item Si la posición de la partícula es incierta, e.d., si
\begin{equation}
  \Delta x>\frac{\hbar }{m_0c},
\end{equation}
entonces el tiempo es tambien incierto, debido a
\begin{equation}
  \Delta t\sim \frac{\Delta x}{c}>\frac{\hbar}{m_0c^2}.
\end{equation}
En una teoría no-relativista $\Delta t$ puede ser arbitrariamente pequeño, porque $c\to \infty$. De este modo, necesitamos la necesidad de reconsiderar el concepto de densidad de probabilidad
\begin{equation}
  \rho(x,y,z,t),
\end{equation}
el cual describe la probabilidad de encontrar una partícula en un lugar definido $\vec{r}$ a un tiempo fijo $t$.
\item A altas energías (relativista) ocurre creación y aniquilación de pares, usualmente en la forma de creación de pares de partícula-antipartícula. Así, a energías relativistas, la conservación de partículas no es más un supuesto válido. Una teoría relativista debe ser capáz de describir creación de pares, polarización del vacio, conversión de partículas, etc. 
\end{enumerate}

\subsection{La ecuación de Klein-Gordon}
De la mecánica cuántica sabemos que la ecuación de Schrödinger
\begin{equation}
  i\hbar \pdv{\psi}{t}=\left[-\frac{\hbar^2}{2m_0}\nabla^2+V(\vec{x})\right]\psi\xt 
\end{equation}
corresponde a la relación de energía no-relativista en forma de operador,
\begin{equation}
  \hat{E}=\frac{\hat{\vec{p}}^2}{2m_0}+V(\vec{x}),\quad \rm donde
\end{equation}
\begin{equation}
  \hat{E}=i\hbar\pdv{t},\qquad \hat{\vec{p}}=-i\hbar\nabla
\end{equation}
son los operadores de energía y momentum respectivamente. Con el fin de obtener una ecuación de onda relativista, comencemos considernado partículas libres con la relación relativista
\begin{equation}
  p^\m p_\m =\frac{E^2}{c^2}-\vec{p}\cdot\vec{p}=m_0^2c^2.
\end{equation}
Reemplazamos el 4-momentum $p^\m $ por el operador 4-momentum
\begin{align}
  \hat{p}^\m =i\hbar\pdv{x_\m }&=i\hbar\left(\pdv{(ct)},\pdv{x_1},\pdv{x_2},\pdv{x_3}\right)\\
  &=i\hbar\left(\pdv{(ct)},-\pdv{x},-\pdv{y},-\pdv{z}\right)\\
  &=i\hbar\left(\pdv{(ct)},-\nabla\right)\\
  &=(\hat{p}_0,\hat{\vec{p}})
\end{align}
Así, obtenemos la \textbf{ecuación de Klein-Gordon} para partículas libres,
\begin{equation}\label{1.21}
  \boxed{\hat{p}^\m \hat{p}_\m \psi=m_0^2c^2\psi}
\end{equation}
Aquí $m_0$ es la masa en reposo de la partícula y $c$ es la velocidad de la luz en el vacío. Notemos que
\begin{align}
  \hat{p}^\m \hat{p}_\m&=\left[i\hbar\left(\pdv{(ct)},-\nabla\right)\right]\left[i\hbar\left(\pdv{(ct)},\nabla\right)\right]\\
  &=-\hbar^2\left[\pdv{(ct)},-\nabla\right]\left[\pdv{(ct)},\nabla\right]\\
  &=-\hbar^2\left(\frac{1}{c^2}\pdv[2]{t},-\nabla^2\right)\\
  &=-\hbar^2\Box
\end{align}
Así, podemos escribir \eqref{1.21} como
\begin{equation}\label{1.22}
  \left(\Box+\frac{m_0^2c^2}{\hbar^2}\right)\psi=0
\end{equation}
Podemos verificar inmediatamente que covaiancia de Lorentz de la ecuación de Klein-Gordon, dado que $ \hat{p}^\m \hat{p}_\m$ es invariante de Lorentz. Notemos también que \eqref{1.22} es la ecuación de onda clásica incluyendo el \textit{término de masa} $m_0^2c^2/\hbar^2$. Soluciones libres son de la forma
\begin{equation}\label{1.23}
  \psi=\exp\left(-\frac{i}{\hbar} p_\m x^\m \right)=\exp\left[-\frac{i}{\hbar}\left(p_0x^0-\vec{p}\cdot\vec{x}\right)\right]=\exp\left[\frac{i}{\hbar}(\vec{p}\cdot\vec{x}-Et)\right]
\end{equation}
En efecto, de \eqref{1.22},
\begin{align}
  \Box\psi+\frac{m_0^2c^2}{\hbar^2}\psi&=\partial^\m\partial_\m e^{-\frac{i}{\hbar}p_\m x^\m }+\frac{m_0^2c^2}{\hbar^2}e^{-\frac{i}{\hbar}p_\m x^\m }\\
  &=\partial^\m \left[-\frac{i}{\hbar}p_\m e^{-\frac{i}{\hbar}p_\m x^\m }\right]+\frac{m_0^2c^2}{\hbar^2}e^{-\frac{i}{\hbar}p_\m x^\m }\\
  &=-\frac{1}{\hbar^2}p_\m p^\m e^{-\frac{i}{\hbar}p_\m x^\m }+\frac{m_0^2c^2}{\hbar^2}e^{-\frac{i}{\hbar}p_\m x^\m }\\
  &=0\qquad \checkmark
\end{align}
bajo la condición $p^\m p_\m =m_0^2c^2$ lo que resuta en
\begin{equation}\label{1.24}
  E=\pm \sqrt{m_0c^2+\vec{p}^2}
\end{equation}



Así, existe solución tanto para energías positivas como para energías negativas \footnote{Las soluciones con eneergía negativa están físicamente conectadas con antipartículas. Dado que estas pueden ser observadas en la naturalez, hemos obtenido una indicación del valor del extender la teoría no-relativista.}.

Ahora constrimos la cuadri-corriente $j_\m $ conectada con \eqref{1.21}. En analogía con nuestas consideraciones para el caso de la ecuación de Schrödinger, esperamos una ley de conservación para $j_\m $. Empezamos de \eqref{1.22}, en la forma
\begin{equation}\label{1.psi*}
  (\hat{p}_\m  \hat{p}^\m -m_0^2c^2)\psi=0\quad /\psi^* \cdot
\end{equation}
y tomamos el comlejo conjugado
\begin{equation}\label{1.psi}
  (\hat{p}_\m  \hat{p}^\m -m_0^2c^2)\psi^*=0\quad /\psi\cdot
\end{equation}
Restando \eqref{1.psi} de \eqref{1.psi*}, es direto ver que
\begin{equation}
  \psi^*\nabla_\m \nabla^\m \psi-\psi\nabla_\m \nabla^\m \psi^*=0
\end{equation}\label{1.25}
\begin{equation}
  \implies \nabla_\m \underbrace{(\psi^*\nabla^\m \psi-\psi\nabla^\m \psi^*)}_{\sim j^\m }=0
\end{equation}
Definimos la cuadri-corriente como
\begin{equation}\label{1.26}
  j_\m \equiv\frac{i\hbar}{2m_0}(\psi^*\nabla^\m \psi-\psi\nabla^\m \psi^*)
\end{equation}
donde agregamos el factor delante para que la componente $j_0$ tenga unidades de probabilidad de densidad (es decir, 1/cm$^3$). Además, esto nos asegura obtener el límite no-relaivista correcto. En particular, \eqref{1.25} queda
\begin{align*}
  \nabla_\m j^\m &=\pdv{(ct)}\left[\frac{i\hbar}{2m_0}\left(\psi^*\pdv{(ct)}\psi-\psi\pdv{(ct)}\psi^*\right)\right]-\nabla\left[\frac{i\hbar}{2m_0}\left(\psi^*\pdv{(ct)}\psi-\psi\pdv{(ct)}\psi^*\right)\right]\\
  &=\pdv{t}\left[\frac{i\hbar}{2m_0c^2}\left(\psi^*\pdv{t}\psi-\psi\pdv{t}\psi^*\right)\right]-\nabla\left[\frac{i\hbar}{2m_0}\left(\psi^*\pdv{(ct)}\psi-\psi\pdv{(ct)}\psi^*\right)\right]\\
  &=\pdv{\rho}{t}+\nabla\cdot\vec{j}\\
  &=0
\end{align*}
es decir, posee la forma de una ecuación de continuidad. Como es usual, integramos sobre todo el espacio,
\begin{equation}
  \int_V\pdv{\rho}{t}\dd^3x=\pdv{t}\int_V\rho\dd^3x=-\int_V\nabla\cdot\vec{j}\dd^3x=-\int_{S}\vec{j}\cdot \dd \vec{S}=0
\end{equation}
\begin{equation}
  \implies \int_V\rho\dd^3x=\rm constante
\end{equation}\label{1.29}
De esta manera, es natural interpretar
\begin{equation}
  \rho =\frac{i\hbar}{2m_0c^2}\left(\psi^*\pdv{t}\psi-\psi\pdv{t}\psi^*\right)
\end{equation}
como una densidad de probabilidad. Sin embargo, hay un problema con esta interpretación: Para un tiempo dado $t$, tanto $\psi$ como $\pdv*{\psi}{t}$ pueden tener valores arbitrarios; luego, $\rho\xt $ en \eqref{1.29} puede ser tanto positiva como negativa, de manera que $\rho\xt $ no está definida positiva y no puede ser considerada una densidad de probabilidad. La razón profunda de esto es que la ecuación de Klein-Gordon es de segundo orden en el tiempo, de manera que debemos conoceer tanto $\psi$ como $\pdv*{\psi\xt }{t}$ para un dado $t$. Además, existen soluciones para energía negativa [ver \eqref{1.24}]. Esto, sumado con la dificultad de la interpretación de la probabilidad fue la razón por la que tanto tiempo se consideró a la ecuación de Klein-Gordon como sin sentido físico. Por lo tanto se intentó buscar una ecuación de onda relativista que fuese de primer orden en el tiempo con densidad de probabilidad definida positiva, la cual fue finalmente descubiera por Dirac. Sin embargo, dicha ecuación sigue teniendo soluciones con energía negativa, las cuales están conectadas con la existencia de antipartículas.

\subsection{El límite no-relativista}
Podemos estudiar el límite no-relativista de la ecuación de Klein-Gordon \eqref{1.21}. Para ello, consideremos el siguiente ansatz
\begin{equation}
  \psi\rt =\varphi\rt e^{-\frac{i}{\hbar}m_0c^2t}
\end{equation}
es decir, separamos la dependencia del tiempo de $\psi$ en dor términos, uno conteniendo la masa en reposo. En el límite no-relativista la diferencia entre la energía total $E$ de la partícula y su masa en reposo $m_0c^2$ es pequeña. Así, definimos
\begin{equation}
  E'=E-m_0c^2
\end{equation}
y remarquemos que la energía $E'$ es no-relativista, lo que significa que $E'\ll m_0c^2$. De aquí,
\begin{equation}
  E'\varphi\ll m_0c^2\varphi\implies \left|i\hbar\pdv{\varphi}{t}\right|\ll m_0c^2\varphi
\end{equation}
Calculemos $\pdv*[2]{\psi}{t}$:
\begin{align}
  \pdv{\psi}{t}&=\pdv{\varphi}{t}e^{-\frac{i}{\hbar}m_0c^2t}-\frac{i m_0c^2}{\hbar}\varphi e^{-\frac{i}{\hbar}m_0c^2t}\\
  &=\left(\pdv{\varphi}{t}-\frac{i m_0c^2}{\hbar}\varphi\right)e^{-\frac{i}{\hbar}m_0c^2t}\\
  &\approx -\frac{i}{\hbar}m_0c^2\varphi e^{-\frac{i}{\hbar}m_0c^2t}
\end{align}
\begin{align}
  \pdv[2]{\psi}{t}&=\pdv{t}\left(\pdv{\varphi}{t}-\frac{i m_0c^2}{\hbar}\varphi\right)e^{-\frac{i}{\hbar}m_0c^2t}\\
  &\approx\left(-\frac{i}{\hbar}m_0c^2\varphi-\frac{i}{\hbar}m_0c^2\varphi\right)e^{-\frac{i}{\hbar}m_0c^2t}\\
  &\approx \left(-2\frac{i}{\hbar}m_0c^2\varphi\right)e^{-\frac{i}{\hbar}m_0c^2t}\\
  &=-\frac{2i m_0c^2}{\hbar}\pdv{\varphi}{t}e^{-\frac{i}{\hbar}m_0c^2t}-\frac{i}{\hbar^2}m_0^2c^4\varphi e^{-\frac{i}{\hbar}m_0c^2t}\\
  &=-\left[\frac{2im_0c^2}{\hbar}\pdv{\varphi}{t}+\frac{m_0^2c^4}{\hbar^2}\varphi\right]e^{-\frac{i}{\hbar}m_0c^2t}
\end{align}
Reemplazando en la ecuación de Klein-Gordon
\begin{equation}
  \hat{p}^\m \hat{p}_\m \psi=m_0c^2c^2\psi\implies \left(\Box+\frac{m_0^2c^2}{\hbar{^2}}\right)\psi=0
\end{equation}
lo cual se puede reescribir como
\begin{equation}
  \left(\frac{1}{c^2}\pdv[2]{t}-\pdv[2]{x}-\pdv[2]{y}-\pdv[2]{z}\right)\psi=0
\end{equation}
se tiene
\begin{align}
  -\frac{1}{c^2}\left[i\frac{2m_0c^2}{\hbar}\pdv{\varphi}{t}+\frac{m_0^2c^4}{\hbar^2}\varphi\right]e^{-\frac{i}{\hbar}m_0c^2t}&=\left(\pdv[2]{x}+\pdv[2]{y}+\pdv[2]{z}-\frac{m_0^2c^2}{\hbar^2}\right)\varphi e^{-\frac{i}{\hbar}m_0c^2t}\\
  -\frac{2im_0}{\hbar}\pdv{\varphi}{t}e^{-\frac{i}{\hbar}m_0c^2t}-\frac{m_0^2c^2}{\hbar^2}\varphi e^{-\frac{i}{\hbar}m_0c^2t}&=\nabla^2\varphi e^{-\frac{i}{\hbar}m_0c^2t}-\frac{m_0^2c^2}{\hbar^2}\varphi e^{-\frac{i}{\hbar}m_0c^2t}\\
  -\frac{2im_0}{\hbar}\pdv{\varphi}{t}e^{-\frac{i}{\hbar}m_0c^2t}&=\nabla^2\varphi e^{-\frac{i}{\hbar}m_0c^2t}
\end{align}
Multiplicando ambos lados por $-\hbar^2/(2m_0)$, obtenemos finalmente
\begin{equation}
\boxed{  i\hbar \pdv{\varphi}{t}=-\frac{\hbar^2}{2m_0}\nabla^2\varphi}
\end{equation}
la ecuación de Schrödinger libre para partículas sin spin. Dado que el tipo de partículas descrito por una ecuación de onda no depende de si la partícula es relativista o no-relativista, inferimos que \textit{la ecuación de Klein-Gordon describe partículas con spin-$0$}.

\subsection{Partículas libres de spin-$0$}
Anteriormente hemos remarcado que en una teoría relativista,\textit{ el concepto de una partícula libre es una idealización}. Es más, partículas de spin-$0$, como piones o kaones, interactúan fuertemente con otras partículas y campos. Sin embargo, podemos descubrir algunos de los métodos prácticos para lidiar con esos problemas, estudiando las soluciones libres de \eqref{1.21}. Volvamos a la interpretación de la densidad de corriente \eqref{1.26} que habíamos descartado debido a que $\rho$ en \eqref{1.29} no era definida positiva. Además, encontramos que, siguiendo el procedimiento usual
\begin{equation}
  \int_V\rho\dd^3x=\rm constante
\end{equation}
La pregunta es cómo se debe interpretar $\rho$ y $\vec{j}$. La interpretación de probabilidad está descartada por lo discutido anteriormente en el contexto de \eqref{1.29}. Sin embargo, tenemos la siguiente alternativa: Obtenemos \textit{la cuadri-corriente densidad de carga} multiplicando la densidad de corriente \eqref{1.26} con la carga elemental $e$, resultando
\begin{align}
  j_\m '\equiv j_\m e=\frac{i e\hbar}{2m_0}(\psi^*\nabla^\m \psi-\psi\nabla^\m \psi^*)=(c\rho',-\vec{j}')
\end{align}
donde
\begin{align}
 \label{1.35} \rho'&=\frac{i\hbar e}{2m_0c^2}\left(\psi^*\pdv{\psi}{t}-\psi\pdv{\psi^*}{t}\right)\\
  \vec{j}'&=-\frac{i\hbar e}{2m_0}(\psi^*\nabla \psi-\psi\nabla\psi^*)\label{1.36}
\end{align}
denotan la \textit{densidad de carga} y la \textit{densidad de corriente de carga}, respectivamente. La densidad de carga \eqref{1.35} puede ser positiva, negativa o cero. Esto equivale a la existencia de partículas y antipartículas en la teoría. Calculando las \textit{soluciones para partículas libres}, podemos entender esto mejor. Consideremos la ecuación de Klein-Gordon \eqref{1.21}
\begin{equation}
  (\hat{p}^\m \hat{p}_\m -m_0^2c^2)\psi=0
\end{equation}
y el ansatz para ondas libres \eqref{1.23}
\begin{equation}
  \psi=Ae^{-\frac{i}{\hbar}p_\m x^\m }=Ae^{\frac{i}{\hbar}(\vec{p}\cdot\vec{x}-Et)}
\end{equation}
vimos que obtenemos como condición necesaria que
\begin{equation}
  p^\m p_\m -m_0^2c^2=0\implies E^2=c^2(\vec{p}^2+m_0^2c^2)
\end{equation}
Así, existen dos posibles soluciones para un momentum $\vec{p}$ dado: una con energía positiva y otra con energía negativa,
\begin{equation}
  E_p=\pm c\sqrt{\vec{p}^2+m_0^2c^2},\qquad \psi_{\pm }=A_{\pm }\exp\left[\frac{i}{\hbar}(\vec{p}\cdot\vec{x}\mp |E_p|t)\right]
\end{equation}
donde los $A_\pm$ son constantes de normalización que serán determinadas más adelante. Reemplazando en la ecuación para la densidad \eqref{1.35}, se tiene
\begin{align}
  \rho'&=\frac{i\hbar e}{2m_0c^2}\left[\psi^*(\mp |E|\psi_{\pm })\frac{i}{\hbar}-\psi\left((\mp |E|\psi^*_{\pm }\left(-\frac{i}{\hbar}\right)\right)\right]\\
  &=\frac{e}{2m_0c^2}\left(\pm |E|\psi^*_{\pm}\psi_{\pm }\pm |E|\psi^*_\pm \psi_\pm \right)\\
  &=\frac{e}{m_0c^2}\left(\pm |E|\psi^*_{\pm}\psi_{\pm }\pm |E|\psi^*_\pm \psi_\pm \right)
\end{align}
\begin{equation}\label{1.39}
  \implies \boxed{\rho_\pm =\pm\frac{e|E_p|}{m_0c^2}\psi^*_\pm \psi_\pm }
\end{equation}
Esto sugiere la siguiente interpretración: $\psi_+$ especifica partículas con carga $+e$; $\psi_-$ especifica partículas con la misma masa, pero con carga $-e$. La solución de la ecuación de onda es siempre una combinación lineal de ambos tipos de funciones. Este punto puede aclararse aún más discretizando las ondas planas continuas \eqref{1.23}. Para ello, confinemos las ondas en una caja de lados $L$. Como es constumbre, imponemos condiciones de borde periódicas en las paredes de la caja. Esto nos lleva a la conocida ecuación
\begin{equation}
  \psi_{n(\pm )}=A_{n(\pm )}\exp\left[\frac{i}{\hbar}(\vec{p}\cdot\vec{x}\mp |E_p|t)\right]
\end{equation}
donde
\begin{equation}
  \vec{p}_n=\frac{2\p }{L},\quad \vec{n}=\{n_x,n_y,n_z\},\quad n_i\in\mathbb{N}
\end{equation}
y
\begin{equation}
  E_{p_n}=c\sqrt{p_n^2+m_0^2c^2}\equiv E_n
\end{equation}

Usando \eqref{1.39} podemos determinar los factores de normalización $A_{\pm }$, pidiendo que
\begin{equation}
  \pm e=\int_{L^3}\dd^3x\rho_\pm =\pm \frac{e E_{p_n}}{m_0c^2}|A_{n(\pm )}|^2L^3
\end{equation}
\begin{equation}
  \implies A_{n(\pm )}=\sqrt{\frac{m_0c^2}{L^3E_{p_n}}}
\end{equation}
Luego,
\begin{equation}\label{1.43}
  \psi_{n(\pm )}=\sqrt{\frac{m_0c^2}{L^3E_{p_n}}}\exp\left[\frac{i}{\hbar}(\vec{p}\cdot\vec{x}\mp |E_p|t)\right]
\end{equation}
Notemos que la normalización para amas soluciones es la misma. La única diferencia es el factor relativo delante de $E_{p_n}t$ en la exponencial. Las soluciones más generales de la ecuación de Klein-Gordon para partículas positivas y negativas de spin-$0$ entonces quedan
\begin{equation}
  \psi_+=\sum_na_n\psi_{n(+)}=\sum_na_n\sqrt{\frac{m_0c^2}{L^3E_{p_n}}}\exp\left[\frac{i}{\hbar}(\vec{p}\cdot\vec{x}- E_nt)\right]
\end{equation}
\begin{equation}
  \psi_-=\sum_nb_n\psi_{n(-)}=\sum_nb_n\sqrt{\frac{m_0c^2}{L^3E_{p_n}}}\exp\left[\frac{i}{\hbar}(\vec{p}\cdot\vec{x}+ E_nt)\right]
\end{equation}
respectivamente. Soluciones para partículas de spin-$0$ con carga nula pueden también ser construidas. Notemos que de la densidad de carga \eqref{1.35}, \textit{el campo de Klein-Gordon $\psi$ tiene que ser real para partículas neutras}, es decir,
\begin{equation}
  \psi=\psi^*
\end{equation}
Por medio de \eqref{1.43} podemos describir fácilmete un frente de onda para una partícula neutra:
\begin{align}
  \psi_{n(0)}&=\frac{1}{\sqrt{2}}\left(\psi^n_{(+)}(\vec{p}_n)+\psi^n_{(-)}(-\vec{p}_n)\right)\\
  &=\sqrt{\frac{m_0c^2}{2L^3E_n}}\left\{\exp\left[\frac{i}{\hbar}(\vec{p}_n\cdot\vec{x}-E_nt)\right]+\exp\left[\frac{i}{\hbar}(-\vec{p}_n\cdot\vec{x}+E_nt)\right]\right\}\\
  &=\sqrt{\frac{m_0c^2}{2L^3E_n}}\left\{\exp\left[\frac{i}{\hbar}(\vec{p}_n\cdot\vec{x}-E_nt)\right]+\exp\left[-\frac{i}{\hbar}(\vec{p}_n\cdot\vec{x}-E_nt)\right]\right\}\\
  &=\sqrt{\frac{m_0c^2}{2L^3E_n}}2\cos\left(\frac{\vec{p}_n\cdot\vec{x}-E_nt}{\hbar}\right)
\end{align}
Notemos que satisface $\psi^n_{(0)}=\psi^{n*}_{(0)}$, luego, de \eqref{1.35} se tiene que $\rho'=0$.

Además, notemos que la densidad de corriente $\vec{j}'\xt $ para partículas neutras \eqref{1.36} se anula también, de manera que en este caso no hay una ley de conservación. Obviamente la teoría cuántica relativista necesariamente conduce a nuevos grados de libertad, esto es, los \textit{grados de libertad de carga} de partículas. En una teoría libre no-relativista, las partículas si spin pueden propagarse libremente con un momentum bien definido $\vec{p}$. En el caso de relativista de una partícula libre y sin spin, \textit{existen tres soluciones, las cuales corresponden a la carga eléctricad e la partículas ($+,-,0$) para cada momentum $\vec{p}$.}






































































