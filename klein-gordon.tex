\section{Ecuación de onda relativista para partículas de spin-$0$. Ecuación de Klein-Gordon}

La descripción de un fenómeno a altas energías requiere la investigación de ecuaciones de ondas relativistas, es decir, \textit{ecuaciones que sean invariantes de Lorentz}. La transición desde una descripción no-relativista hacia una relativista implica que varios conceptos de la teoría no-relativista deben ser reinvestigados, en particular:
\begin{enumerate}
	\item Coordenadas espaciales y temporales tienen que ser tratadas por igual en la teoría.
	\item Dado que
	\begin{equation}
  \Delta x\sim \frac{\hbar}{\Delta p}\sim \frac{\hbar }{m_0c},
\end{equation}
una partícula relativista no puede ser localizada con más precisión que $\approx \hbar/m_0c$; de otro modo podría ocurrir creación de pares de partículas debido a $E>2m_0c^2$. Así, la idea de partícula libre solo hace sentido si la partícula no está confinada por vínculos externos a un volúmen menor que aproximadamente la \textit{longitud de onda de Compton} $\lambda_{ c}=\hbar/m_0c $. De otro modo, la partícula automáticamente tiene compañeros debido a la reación de partícula-antipartícula.
\item Si la posición de la partícula es incierta, e.d., si
\begin{equation}
  \Delta x>\frac{\hbar }{m_0c},
\end{equation}
entonces el tiempo es tambien incierto, debido a
\begin{equation}
  \Delta t\sim \frac{\Delta x}{c}>\frac{\hbar}{m_0c^2}.
\end{equation}
En una teoría no-relativista $\Delta t$ puede ser arbitrariamente pequeño, porque $c\to \infty$. De este modo, necesitamos la necesidad de reconsiderar el concepto de densidad de probabilidad
\begin{equation}
  \rho(x,y,z,t),
\end{equation}
el cual describe la probabilidad de encontrar una partícula en un lugar definido $\vec{r}$ a un tiempo fijo $t$.
\item A altas energías (relativista) ocurre creación y aniquilación de pares, usualmente en la forma de creación de pares de partícula-antipartícula. Así, a energías relativistas, la conservación de partículas no es más un supuesto válido. Una teoría relativista debe ser capáz de describir creación de pares, polarización del vacio, conversión de partículas, etc. 
\end{enumerate}

\subsection{La ecuación de Klein-Gordon}
De la mecánica cuántica sabemos que la ecuación de Schrödinger
\begin{equation}
  i\hbar \pdv{\psi}{t}=\left[-\frac{\hbar^2}{2m_0}\nabla^2+V(\vec{x})\right]\psi\xt 
\end{equation}
corresponde a la relación de energía no-relativista en forma de operador,
\begin{equation}
  \hat{E}=\frac{\hat{\vec{p}}^2}{2m_0}+V(\vec{x}),\quad \rm donde
\end{equation}
\begin{equation}
  \hat{E}=i\hbar\pdv{t},\qquad \hat{\vec{p}}=-i\hbar\nabla
\end{equation}
son los operadores de energía y momentum respectivamente. Con el fin de obtener una ecuación de onda relativista, comencemos considernado partículas libres con la relación relativista
\begin{equation}
  p^\m p_\m =\frac{E^2}{c^2}-\vec{p}\cdot\vec{p}=m_0^2c^2.
\end{equation}
Reemplazamos el 4-momentum $p^\m $ por el operador 4-momentum
\begin{align}
  \hat{p}^\m =i\hbar\pdv{x_\m }&=i\hbar\left(\pdv{(ct)},\pdv{x_1},\pdv{x_2},\pdv{x_3}\right)\\
  &=i\hbar\left(\pdv{(ct)},-\pdv{x},-\pdv{y},-\pdv{z}\right)\\
  &=i\hbar\left(\pdv{(ct)},-\nabla\right)\\
  &=(\hat{p}_0,\hat{\vec{p}})
\end{align}
Así, obtenemos la \textbf{ecuación de Klein-Gordon} para partículas libres,
\begin{equation}\label{1.21}
  \boxed{\hat{p}^\m \hat{p}_\m \psi=m_0^2c^2\psi}
\end{equation}
Aquí $m_0$ es la masa en reposo de la partícula y $c$ es la velocidad de la luz en el vacío. Notemos que
\begin{align}
  \hat{p}^\m \hat{p}_\m&=\left[i\hbar\left(\pdv{(ct)},-\nabla\right)\right]\left[i\hbar\left(\pdv{(ct)},\nabla\right)\right]\\
  &=-\hbar^2\left[\pdv{(ct)},-\nabla\right]\left[\pdv{(ct)},\nabla\right]\\
  &=-\hbar^2\left(\frac{1}{c^2}\pdv[2]{t},-\nabla^2\right)\\
  &=-\hbar^2\Box
\end{align}
Así, podemos escribir \eqref{1.21} como
\begin{equation}\label{1.22}
  \left(\Box+\frac{m_0^2c^2}{\hbar^2}\right)\psi=0
\end{equation}
Podemos verificar inmediatamente que covaiancia de Lorentz de la ecuación de Klein-Gordon, dado que $ \hat{p}^\m \hat{p}_\m$ es invariante de Lorentz. Notemos también que \eqref{1.22} es la ecuación de onda clásica incluyendo el \textit{término de masa} $m_0^2c^2/\hbar^2$. Soluciones libres son de la forma
\begin{equation}
  \psi=\exp\left(-i\hbar p_\m x^\m \right)=\exp\left[-\frac{i}{\hbar}\left(p_0x^0-\vec{p}\cdot\vec{x}\right)\right]=\exp\left[\frac{i}{\hbar}(\vec{p}\cdot\vec{x}-Et)\right]
\end{equation}






































